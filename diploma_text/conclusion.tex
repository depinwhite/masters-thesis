\newpage

\section{Заключение}

В данной работе рассматриваются два взаимодополняющих исследования, посвящённых задаче исправления грамматических ошибок, с фокусом на морфологически сложные языки с ограниченными ресурсами. В отличие от английского языка, для которого данная задача изучена достаточно хорошо, языки с богатой морфологией, такие как русский, остаются недостаточно исследованными из-за дефицита размеченных данных и сложности коррекции ошибок в подобных языковых контекстах.

В первом исследовании представлена интерпретируемая и эффективная система исправления ошибок в русских текстах, основанная на правилах и синтетическом наборе обучающих данных. Используя эффективную архитектуру GECToR, система продемонстрировала высокую точность на синтетических данных $F_{0.5}$ = 82,5 и показала способность к обобщению на реальных текстах $F_{0.5}$ = 22,2, превзойдя базовые модели, не обученные напрямую на таких данных. Перспективы улучшения включают расширение словаря корректирующих правил, дообучение на дополнительных датасетах и эксперименты с различными архитектурами для повышения качества исправлений.

Второе исследование предлагает полностью неконтролируемый подход к исправлению грамматических ошибок, не требующий ручной разметки данных или языково-специфичных правил. Метод основан на базовых преобразованиях на уровне подслов и использовании выравнивания по Левенштейну для генерации исправлений. Продемонстрировав эффективность для английского языка, $F_{0.5}$ = 62,4 на CoNLL-2014 и $F_{0.5}$ = 61,9 на BEA-2019, подход показал свою конкурентоспособность. Его применение особенно перспективно для малоресурсных языков, где ощущается острый дефицит размеченных данных. Дальнейшие исследования могут быть направлены на адаптацию метода к другим языкам, а также на изучение влияния различных архитектур “Трансформер” и стратегий токенизации на качество исправлений.

Вместе эти исследования демонстрируют перспективность как гибридных (основанных на правилах), так и полностью неконтролируемых подходов для решения задачи исправления грамматических ошибок в условиях ограниченных языковых ресурсов. Если первая работа подтверждает эффективность синтетических данных и структурированных корректирующих правил, то вторая доказывает возможность достижения конкурентоспособных результатов без использования размеченных данных. 
