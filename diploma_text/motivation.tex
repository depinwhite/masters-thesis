\newpage

\section{Универсальный подход для исправления грамматических ошибок на уровне подслов}
\subsection{Мотивация перехода на уровень подслов}

Токенизация является важнейшим этапом вычислительной обработки текста~\cite{b29,b30}, поскольку позволяет сократить размер словаря и повысить эффективность работы модели. В зависимости от архитектуры модели применялись различные методы токенизации: Byte Pair Encoding (BPE) для RoBERTa, WordPiece для BERT и SentencePiece для XLNet. В данном исследовании под “подсловом” понимается минимальная единица текста, полученная в результате разбиения строки символов с помощью токенизатора соответствующей модели на основе архитектуры “Трансформер”.

В подходе~\cite{b15}, основанном на ST-модели, словарь корректирующих правил формируется на уровне целых слов. Таким образом, во время обучения и предсказания все подслова, входящие в состав одного слова, ассоциируются с единым правилом исправления из словаря.

В работе~\cite{b15} вводится разделение преобразований на базовые и грамматические (g-transformations). Базовые преобразования ограничиваются простейшими операциями редактирования --- удалением, сохранением, заменой и добавлением слов. В отличие от них, грамматические преобразования требуют глубокого лингвистического анализа и включают такие операции, как изменение глагольных форм, коррекцию числа и падежа существительных, а также слияние или разделение слов.

Реализация грамматических преобразований для произвольного языка ставит три ключевые задачи: формализацию грамматических правил языка, наличие размеченных обучающих данных и разработку механизма сопоставления экспертных меток с корректирующими правилами. На уровне подслов нет необходимости разрабатывать специфические для задачи операции, поскольку каждое слово может быть исправлено конечным числом “базовых преобразований” без полного изменения слова.

Это ключевой момент, так как задача может быть решена с помощью обучения без учителя:
\begin{enumerate}
    \item Отсутствие необходимости в размеченных данных: разметка на уровне подслов автоматически генерируется при обратном проходе по матрице расстояний, полученных с помощью алгоритма Левенштейна. Данный алгоритм поддерживает все базовые операции редактирования --- удаление, сохранение без изменения, замену и добавление. Для работы метода достаточно наличия параллельных текстов --- исходные предложения с ошибками и их исправленные версии.
    \item Отсутствие необходимости в ручной разработке корректирующих правил: все корректирующие правила автоматически извлекаются из базовых преобразований, применяемых на уровне подслов, что устраняет необходимость трудоемкой разработки словарей грамматических правил.
\end{enumerate}

В данной работе предложен метод исправления грамматических ошибок на уровне подслов, который можно обобщить для различных языков без необходимости лингвистического анализа грамматических правил или доступа к размеченным данным.



