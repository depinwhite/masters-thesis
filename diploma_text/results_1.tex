\subsection{Результаты}


\begin{table}[ht!]
\begin{center}
\begin{tabular}{|l|l|l|l|l|}
  \hline
  System & Training stage & $P$ & $R$ & $ F_{0.5} $\\ \hline
  RuGECToR & \RomanNumeralCaps{1} & 88.4 & 67.1 & 83.1 \\ \hline
  RuGECToR & \RomanNumeralCaps{2} & 88.5 & 65.1 & 82.5 \\ \hline
\end{tabular}
\caption{Качество работы модели на синтетическом тестовом наборе данных}
\label{tab3}
\end{center}
\end{table}

Для синтетического тестового набора данных сгенерировано 10.000 предложений с ошибками. Результаты на синтетическом наборе данных приведены в табл.~\ref{tab3}. Видно, что метрики $R$ и $F_{0.5}$ на втором этапе обучения ниже, чем на первом. Это можно объяснить двумя причинами:
\begin{enumerate}
    \item 50\% предложений в данных, использованных для второго этапа обучения, не содержали ошибок. Точность стала выше, но при этом количество охватываемых токенов уменьшилось.
    \item На втором этапе обучения были использованы дополнительные предложения из литературных произведений, которые не использовались во время первого этапа обучения. Таким образом, была увеличена обобщающая способность модели за счет добавления данных из другого распределения.
\end{enumerate}

\begin{table}[ht!]
\begin{center}
\begin{tabular}{|l|l|l|l|l|}
  \hline
  модель & данные для обучения data & $P$ & $R$ & $ F_{0.5} $\\ \hline
  Classifiers (learner) & RULEC & 22.6 & 4.8 & 12.9 \\ \hline
  Classifiers (min sup.) & RULEC & 38.0 & 7.5 & 21.0 \\ \hline
  MT & RULEC & 30.6 & 2.9 & 10.6 \\ \hline
  RuGECToR & Synthetic (\RomanNumeralCaps{1} stage) & 23.6 & 5.6 & 14.3 \\ \hline
  RuGECToR & Synthetic (\RomanNumeralCaps{2} stages) & \textbf{40.8} & \textbf{7.9} & \textbf{22.2} \\ \hline
\end{tabular}
\caption{Сравнение качества работы моделей на наборе данных RULEC}
\label{tab4}
\end{center}
\end{table}

В качестве реальных тестовых данных использован набор данных RULEC. Результаты приведены в табл.~\ref{tab4}. Сравнение производилось с моделями Classifiers (learner), Classifiers (minimal sup.) и MT. Данные модели были представлены в~\cite{b12} и обучены на наборе данных RULEC. Модель RuGECToR достигает значения равного 22.2 c точки зрения метрики $F_{0.5}$ на наборе данных RULEC. Это значение выше результата, демонстрируемого вышеперечисленными моделями, несмотря на то, что модель RuGECToR на нем не обучалась. Таким образом, модель обладает хорошей обобщающей способностью и не переобучается под конкретный набор данных. Следует также отметить, что на синтетических тестовых данных наша модель работает лучше, чем на реальных данных. Это вполне ожидаемо, так как синтетический обучающий и синтетический тестовый наборы данных имеют схожие распределения, в то время как RULEC сильно отличается от синтетического обучающего набора данных. Табл.~\ref{tab4} также показывает, что обобщающая способность модели на втором этапе обучения растет. На наборе данных RULEC качество работы модели после второго этапа значительно выше, чем после первого.
