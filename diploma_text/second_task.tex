\subsection{Обновленная постановка задачи}

Обозначим множество исходных последовательностей как
\[
\mathcal{X} = \{s_i \mid s_i = [x_1, x_2, \ldots, x_{n_i}]\}_{i = 0}^N,
\]
где каждая исходная последовательность $s_i$, разбитая с помощью токенизатора, представлена в виде последовательности подслов длины $n_i$, а $N$ — общее количество предложений.

Множество целевых последовательностей с токенизацией длины $m_i$ аналогично задаётся как
\[
\mathcal{Y} = \{t_i \mid t_i = [y_1, y_2, \ldots, y_{m_i}]\}_{i = 0}^N.
\]

Пусть задан словарь корректирующих правил $W$, содержащий правила: \textit{добавить} (append), \textit{сохранить} (keep), \textit{заменить} (replace) и \textit{удалить} (delete).

Требуется найти множество всех возможных последовательностей корректирующих преобразований с минимальным числом операций вставки, удаления и замены:
\[
\mathcal{F} = \{ w_{ij} \mid w_{ij} \circ s_i \rightarrow t_i \}_{i = 0}^N,
\]
где последовательность корректирующих преобразований из словаря $W$ задаётся как
\[
w_{ij} = \{w_1^*, w_2^*, \ldots, w_{n_i}^*\}_{j}, j \in \{0, \ldots, o_i\},
\]
где $o_i$ — количество последовательностей корректирующих преобразований минимальной длины, преобразующих $s_i$ в $t_i$.

Операция $\circ$ обозначает покомпонентное применение правил коррекции к подсловам.
