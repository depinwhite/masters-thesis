\newpage


\begin{abstract}
В данной работе представлены два исследования, посвящённые задаче исправления грамматических ошибок в текстах с использованием подхода Sequence Tagging. В первом исследовании описывается адаптация модели GECToR для русского языка. С учетом недостатка размеченных данных, для обучения модели был создан синтетический набор данных. Разработанная модель показала хорошие результаты на синтетических данных $F_{0.5}$ = 82.5, а также продемонстрировала способность к переносу знаний на тестовый набор данных RULEC без дополнительного обучения $F_{0.5}$ = 22.2.

Во втором исследовании предлагается полностью автоматизированный, не требующий разметки подход к решению задачи исправления грамматических ошибок. Метод основан на генерации данных с использованием алгоритма Левенштейна для исправления грамматических ошибок на уровне подслов с использованием правил: keep, append, replace и delete. Подход универсален для любого языка и не требует дополнительной разметки. Применение данного метода к оригинальной модели GECToR позволило достичь конкурентных результатов на английском языке: $F_{0.5}$ = 62.4 на CoNLL-2014 и $F_{0.5}$ = 61.9 на BEA-2019, при этом не потребовалось ни аннотирования данных, ни составления словаря грамматических правил.

Таким образом, совместное рассмотрение обоих исследований демонстрирует возможности применения и адаптации Sequence Tag-ging моделей как для языков с достаточным количеством размеченных данных, так и для языков, где количество таких данных ограничено.

\smallskip
\textbf{Ключевые слова}: исправление грамматических ошибок; обработка естественного языка; алгоритм левенштейна; трансформер.
\end{abstract}




