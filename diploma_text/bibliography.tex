\newpage


\begin{thebibliography}{99}

\bibitem{b1}
Rozovskaya A., Roth D. Grammatical Error Correction: Machine Translation and Classifiers // Proceedings of the 54th Annual Meeting of the Association for Computational Linguistics (Volume 1: Long Papers). — Berlin, Germany: Association for Computational Linguistics, 2016. — P. 2205–2215.

\bibitem{b2}
Yuan Z., Stahlberg F., Rei M., Byrne B., Yannakoudakis H. Neural and FST-based approaches to grammatical error correction // Proceedings of the Fourteenth Workshop on Innovative Use of NLP for Building Educational Applications. — Florence, Italy: Association for Computational Linguistics, 2019. — P. 228–239.

\bibitem{b3}
Bryant C., Ng H.T. How Far are We from Fully Automatic High Quality Grammatical Error Correction? // Proceedings of the 53rd Annual Meeting of the Association for Computational Linguistics. — Beijing, China: Association for Computational Linguistics, 2015. — P. 697–707.

\bibitem{b4}
Rajput D. Review on recent developments in frequent itemset based document clustering, its research trends and applications // International Journal of Data Analysis Techniques and Strategies. — 2019. — Vol. 11. — P. 176–195.

\bibitem{b5}
Flickinger D., Yu J. Toward More Precision in Correction of Grammatical Errors // Proceedings of the Seventeenth Conference on Computational Natural Language Learning: Shared Task. — Sofia, Bulgaria: Association for Computational Linguistics, 2013. — P. 68–73.

\bibitem{b6}
Yuan X., Pham D., Davidson S., Yu Z. ErAConD: Error Annotated Conversational Dialog Dataset for Grammatical Error Correction // Proceedings of the 2022 Conference of the North American Chapter of the Association for Computational Linguistics: Human Language Technologies. — Seattle, USA: Association for Computational Linguistics, 2022. — P. 76–84.

\bibitem{b7}
Lee J.S.Y., Seneff S. An analysis of grammatical errors in non-native speech in English // 2008 IEEE Spoken Language Technology Workshop. — 2008. — P. 89–92.

\bibitem{b8}
Zhuravlev K., Rudakov K., Inyakin A., et al. The system of recognition of intellectual text reuse “Antiplagiat” // Mathematical methods of pattern recognition: 12th All-Russian conference: Collection of reports. — Moscow: MAKS Press, 2005. — P. 329–332.

\bibitem{b9}
Keck C.M. How Do University Students Attempt to Avoid Plagiarism? A Grammatical Analysis of Undergraduate Paraphrasing Strategies // Writing \& Pedagogy. — 2010. — Vol. 2. — P. 193–222.

\bibitem{b10}
Zhang W.E., Sheng Q.Z., Alhazmi A., Li C. Adversarial Attacks on Deep-Learning Models in Natural Language Processing: A Survey // ACM Transactions on Intelligent Systems and Technology. — 2020. — Vol. 11, No. 3. — Art. 24. — P. 1–41.

\bibitem{b11}
Ng H.T., Wu S.M., Briscoe T., et al. The CoNLL-2014 Shared Task on Grammatical Error Correction // Proceedings of the Eighteenth Conference on Computational Natural Language Learning: Shared Task. — Baltimore, MD: Association for Computational Linguistics, 2014. — P. 1–14.

\bibitem{b12}
Rozovskaya A., Roth D. Grammar Error Correction in Morphologically Rich Languages: The Case of Russian // Transactions of the Association for Computational Linguistics. — 2019. — Vol. 7. — P. 1–17.

\bibitem{b13}
Rothe S., Mallinson J., Malmi E., Krause S., Severyn A. A Simple Recipe for Multilingual Grammatical Error Correction // Proceedings of the 59th Annual Meeting of the Association for Computational Linguistics and the 11th International Joint Conference on Natural Language Processing (Volume 2: Short Papers). — Online: Association for Computational Linguistics, 2021. — P. 702–707.

\bibitem{b14}
Grundkiewicz R., Junczys-Dowmunt M., Heafield K. Neural Grammatical Error Correction Systems with Unsupervised Pre-training on Synthetic Data // Proceedings of the Fourteenth Workshop on Innovative Use of NLP for Building Educational Applications. — Florence, Italy: Association for Computational Linguistics, 2019. — P. 252–263.

\bibitem{b15}
Omelianchuk K., Atrasevych V., Chernodub A., Skurzhanskyi O. GECToR – Grammatical Error Correction: Tag, Not Rewrite // Proceedings of the Fifteenth Workshop on Innovative Use of NLP for Building Educational Applications. — Seattle, WA, USA → Online: Association for Computational Linguistics, 2020. — P. 163–170.

\bibitem{b16}
Malmi E., Krause S., Rothe S., Mirylenka D., Severyn A. Encode, Tag, Realize: High-Precision Text Editing // Proceedings of the 2019 Conference on Empirical Methods in Natural Language Processing and the 9th International Joint Conference on Natural Language Processing (EMNLP-IJCNLP). — Hong Kong: Association for Computational Linguistics, 2019. — P. 5054–5065.

\bibitem{b17}
Vaswani A., Shazeer N., Parmar N., et al. Attention is All You Need // Advances in Neural Information Processing Systems. — 2017. — Vol. 30.


\bibitem{b18}
Devlin J., Chang M.-W., Lee K., Toutanova K.  
BERT: Pre-training of Deep Bidirectional Transformers for Language Understanding // Proceedings of the 2019 Conference of the North American Chapter of the Association for Computational Linguistics: Human Language Technologies, Volume 1 (Long and Short Papers). — Minneapolis, Minnesota: Association for Computational Linguistics, 2019. — P. 4171–4186. — DOI: \url{10.18653/v1/N19-1423}.

\bibitem{b19}
Liu Y., Ott M., Goyal N., Du J., Joshi M., Chen D., Levy O., Lewis M., Zettlemoyer L., Stoyanov V.  
RoBERTa: A Robustly Optimized BERT Pretraining Approach [Электронный ресурс]. — 2019. — Режим доступа: \url{http://arxiv.org/abs/1907.11692} (дата обращения: 03.06.2025)


\bibitem{b20}
Yang Z., Dai Z., Yang Y., Carbonell J., Salakhutdinov R., Le Q. V.  
XLNet: generalized autoregressive pretraining for language understanding // Proceedings of the 33rd International Conference on Neural Information Processing Systems. — Red Hook, NY, USA: Curran Associates Inc., 2019. — Article No. 517. — 11 p.

\bibitem{b21}
Khabutdinov I. A., Chashchin A. V., Grabovoy A. V., Kildyakov A. S., Chekhovich U. V.  
RuGECToR: Rule-Based Neural Network Model for Russian Language Grammatical Error Correction // Program. Comput. Softw. — 2024. — Vol. 50, no. 4. — P. 315–321. — DOI: \href{https://doi.org/10.1134/S0361768824700129}{10.1134/S0361768824700129}.

\bibitem{b22}
Kudo T., Richardson J.  
SentencePiece: A simple and language independent subword tokenizer and detokenizer for Neural Text Processing // In: Proceedings of the 2018 Conference on Empirical Methods in Natural Language Processing: System Demonstrations. — Brussels, Belgium: Association for Computational Linguistics, 2018. — P. 66–71. — DOI: \href{https://doi.org/10.18653/v1/D18-2012}{10.18653/v1/D18-2012}.

\bibitem{b23}
Korobov M. Morphological Analyzer and Generator for Russian and Ukrainian Languages // Analysis of Images, Social Networks and Texts. — Springer, 2015. — Vol. 542. — P. 320–332.

\bibitem{b24}
Open source collection of school essays [Электронный ресурс]. — Режим доступа: \url{https://www.kritika24.ru} (дата обращения: 07.11.2022).

\bibitem{b25}
Open source collection of literary works [Электронный ресурс]. — Режим доступа: \url{https://proza.ru} (дата обращения: 07.11.2022).

\bibitem{b26}
Trinh V.A., Rozovskaya A. New Dataset and Strong Baselines for the Grammatical Error Correction of Russian // Findings of the Association for Computational Linguistics: ACL-IJCNLP 2021. — Online: Association for Computational Linguistics, 2021. — P. 4103–4111.

\bibitem{b27}
Lyashevskaya O., Sharov S. Frequency Dictionary of the Modern Russian Language (based on the materials of the National Corpus of the Russian Language) [in Russian]. — Moscow: Azbukovnik, 2009. — 21 p.

\bibitem{b28}
Kingma D., Ba J.  
Adam: A Method for Stochastic Optimization // International Conference on Learning Representations (ICLR). — 2015. — arXiv:1412.6980.

\bibitem{b29}
Limisiewicz T., Balhar J., Mareček D.  
Tokenization Impacts Multilingual Language Modeling: Assessing Vocabulary Allocation and Overlap Across Languages // *Findings of the Association for Computational Linguistics: ACL 2023*. — Toronto, Canada: Association for Computational Linguistics, 2023. — P. 5661–5681. — DOI: 10.18653/v1/2023.findings-acl.350.

\bibitem{b30}
Asvarov A., Grabovoy A.  
The Impact of Multilinguality and Tokenization on Statistical Machine Translation // *Proceedings of the 2024 Conference*. — 2024. — P. 149–157. — DOI: 10.23919/FRUCT61870.2024.10516416.

\bibitem{b31}
Dahlmeier D., Ng H.T.  
Better Evaluation for Grammatical Error Correction // *Proceedings of the 2012 Conference of the North American Chapter of the Association for Computational Linguistics: Human Language Technologies*. — Montréal, Canada: Association for Computational Linguistics, 2012. — P. 568–572.

\bibitem{b32}
Awasthi A., Sarawagi S., Goyal R., Ghosh S., Piratla V.  
Parallel Iterative Edit Models for Local Sequence Transduction // *Proceedings of the 2019 Conference on Empirical Methods in Natural Language Processing and the 9th International Joint Conference on Natural Language Processing (EMNLP-IJCNLP)*. — Hong Kong, China: Association for Computational Linguistics, 2019. — P. 4259–4269. — DOI: 10.18653/v1/D19-1435.

\bibitem{b33}
Mizumoto T., Komachi M., Nagata M., Matsumoto Y.  
Mining revision log of language learning SNS for automated Japanese error correction of second language learners // *Proceedings of the 5th International Joint Conference on Natural Language Processing*. — 2011. — P. 147–155.


\bibitem{b34}
Rothe S., Mallinson J., Malmi E., Krause S., Severyn A.  
A Simple Recipe for Multilingual Grammatical Error Correction // *Proceedings of the 59th Annual Meeting of the Association for Computational Linguistics and the 11th International Joint Conference on Natural Language Processing (Volume 2: Short Papers)*. — 2021. — P. 702–707. — DOI: 10.18653/v1/2021.acl-short.89.

\bibitem{b35}
Dahlmeier D., Ng H. T., Wu S.  
Building a Large Annotated Corpus of Learner English: The NUS Corpus of Learner English~//  
\textit{Proceedings of the Eighth Workshop on Innovative Use of NLP for Building Educational Applications}. — Atlanta, Georgia: Association for Computational Linguistics, 2013. — P.~22--31. — URL: \href{https://aclanthology.org/W13-1703}{aclanthology.org/W13-1703}.

\bibitem{b36}
Yannakoudakis H., Briscoe T., Medlock B.  
A New Dataset and Method for Automatically Grading ESOL Texts~//  
\textit{Proceedings of the 49th Annual Meeting of the Association for Computational Linguistics: Human Language Technologies}. — Portland, Oregon, USA: Association for Computational Linguistics, 2011. — P.~180--189. — URL: \href{https://aclanthology.org/P11-1019}{aclanthology.org/P11-1019}.

\bibitem{b37}
Bryant C., Felice M., Andersen {\O}.E., Briscoe T.  
The BEA-2019 Shared Task on Grammatical Error Correction~//  
\textit{Proceedings of the Fourteenth Workshop on Innovative Use of NLP for Building Educational Applications}. — Florence, Italy: Association for Computational Linguistics, 2019. — P.~52--75.  
DOI: \href{https://doi.org/10.18653/v1/W19-4406}{10.18653/v1/W19-4406}.  
URL: \href{https://aclanthology.org/W19-4406}{aclanthology.org/W19-4406}.



\end{thebibliography}




