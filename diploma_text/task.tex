\newpage

\section{Адаптация модели GECToR для русского языка}
\subsection{Постановка задачи}

Задано множество пар, в котором каждая пара состоит из предложения \( x_i \) и соответствующего ему эталонного исправления \( t_i \):
\[
S = \{(x_i, t_i)\}_{i=1}^N,
\]
где \( N \) – количество пар. 

В данной работе предполагается, что каждое предложение представлено в виде последовательности токенов \( x_i = \{s_1, s_2, \dots, s_{n_i}\} \), а каждое эталонное исправление – в виде последовательности корректирующих правил \( t_i = \{t_1, t_2, \dots, t_{n_i}\} \), где \( n_i \) – длина \( i \)-го предложения. 

Под понятием <<токен>> подразумевается слово, таким образом разбиение предложения на токены происходит на уровне слов. Каждое корректирующее правило \( t_j \) является элементом заданного нами словаря. Составление словаря описано ниже.

Целью задачи является нахождение функции \( T \) для построения отображения из последовательности токенов \( x_i \) в последовательность корректирующих правил \( t_i \):
\[
T: x_i \mapsto t_i \in \{0, 1, \dots, k\}^{n_i},
\]
где \( k \) – размер словаря корректирующих правил.
 

